%%%%%%%%%%%%%%%%%%%%%%%%%%%%%%%%%%%%%%%%%
% Journal Article
% LaTeX Template
% Version 1.4 (15/5/16)
%
% This template has been downloaded from:
% http://www.LaTeXTemplates.com
%
% Original author:
% Frits Wenneker (http://www.howtotex.com) with extensive modifications by
% Vel (vel@LaTeXTemplates.com)
%
% License:
% CC BY-NC-SA 3.0 (http://creativecommons.org/licenses/by-nc-sa/3.0/)
%
%%%%%%%%%%%%%%%%%%%%%%%%%%%%%%%%%%%%%%%%%

%----------------------------------------------------------------------------------------
%	PACKAGES AND OTHER DOCUMENT CONFIGURATIONS
%----------------------------------------------------------------------------------------

\documentclass[twoside,twocolumn]{article}

    \usepackage{blindtext} % Package to generate dummy text throughout this template 
    
    \usepackage[sc]{mathpazo} % Use the Palatino font
    \usepackage[T1]{fontenc} % Use 8-bit encoding that has 256 glyphs
    \linespread{1.05} % Line spacing - Palatino needs more space between lines
    \usepackage{microtype} % Slightly tweak font spacing for aesthetics
    
    \usepackage[english]{babel} % Language hyphenation and typographical rules
    
    \usepackage[hmarginratio=1:1,top=32mm,columnsep=20pt]{geometry} % Document margins
    \usepackage[hang, small,labelfont=bf,up,textfont=it,up]{caption} % Custom captions under/above floats in tables or figures
    \usepackage{booktabs} % Horizontal rules in tables
    
    \usepackage{lettrine} % The lettrine is the first enlarged letter at the beginning of the text
    \usepackage{amsmath} % The lettrine is the first enlarged letter at the beginning of the text

    \usepackage{enumitem} % Customized lists
    \setlist[itemize]{noitemsep} % Make itemize lists more compact
    
    \usepackage{abstract} % Allows abstract customization
    \renewcommand{\abstractnamefont}{\normalfont\bfseries} % Set the "Abstract" text to bold
    \renewcommand{\abstracttextfont}{\normalfont\small\itshape} % Set the abstract itself to small italic text
    
    \usepackage{bm} % Allows customization of titles

    \usepackage{titlesec} % Allows customization of titles
    \renewcommand\thesection{\Roman{section}} % Roman numerals for the sections
    \renewcommand\thesubsection{\roman{subsection}} % roman numerals for subsections
    \titleformat{\section}[block]{\large\scshape\centering}{\thesection.}{1em}{} % Change the look of the section titles
    \titleformat{\subsection}[block]{\large}{\thesubsection.}{1em}{} % Change the look of the section titles
    
    \usepackage{fancyhdr} % Headers and footers
    \pagestyle{fancy} % All pages have headers and footers
    \fancyhead{} % Blank out the default header
    \fancyfoot{} % Blank out the default footer
    \fancyhead[C]{Running title $\bullet$ May 2016 $\bullet$ Vol. XXI, No. 1} % Custom header text
    \fancyfoot[RO,LE]{\thepage} % Custom footer text
    
    \usepackage{titling} % Customizing the title section
    
    \usepackage{hyperref} % For hyperlinks in the PDF
    
    %----------------------------------------------------------------------------------------
    %	FORMULAS?
    %----------------------------------------------------------------------------------------

    \def \faradaysLawDiff {\bm{\nabla \times E} = -\frac{\partial{B}}{\partial{t}}}
    \def \amperesLawDiff {\bm{\nabla \times H} = \bm{J_s}+\frac{\partial{D}}{\partial{t}}}
    \def \gaussLawEDiff {\bm{\nabla \cdot D} = \bm{\rho}}
    \def \gaussLawMDiff {\bm{\nabla \cdot B} = {0}}

    %----------------------------------------------------------------------------------------
    %	TITLE SECTION
    %----------------------------------------------------------------------------------------
    
    \setlength{\droptitle}{-4\baselineskip} % Move the title up
    
    \pretitle{\begin{center}\Huge\bfseries} % Article title formatting
    \posttitle{\end{center}} % Article title closing formatting
    \title{E\&M 101} % Article title
    \author{%
    \textsc{John Smith}\thanks{A thank you or further information} \\[1ex] % Your name
    \normalsize University of California \\ % Your institution
    \normalsize \href{mailto:john@smith.com}{john@smith.com} % Your email address
    %\and % Uncomment if 2 authors are required, duplicate these 4 lines if more
    %\textsc{Jane Smith}\thanks{Corresponding author} \\[1ex] % Second author's name
    %\normalsize University of Utah \\ % Second author's institution
    %\normalsize \href{mailto:jane@smith.com}{jane@smith.com} % Second author's email address
    }
    \date{\today} % Leave empty to omit a date
    \renewcommand{\maketitlehookd}{%
    \begin{abstract}
    \noindent \blindtext % Dummy abstract text - replace \blindtext with your abstract text
    \end{abstract}
    }
    
    %----------------------------------------------------------------------------------------
    
    \begin{document}
    
    % Print the title
    \maketitle
    
    \section{Basic Math}

    \begin{tabular}{ l | c }
        \hline
        \hline			
        $\nabla \times F$ & $\begin{bmatrix}
            x_{11}       & x_{12} & x_{13} & \dots & x_{1n} \\
            x_{21}       & x_{22} & x_{23} & \dots & x_{2n} \\
            \hdotsfor{5} \\
            x_{d1}       & x_{d2} & x_{d3} & \dots & x_{dn}
        \end{bmatrix}$\\
        Ampere's Law & ${\amperesLawDiff}$\\
        
        Gauss' Law, Electric & ${\gaussLawEDiff}$\\
        Gauss' Law, Magnetic & ${\gaussLawMDiff}$\\
        \hline			

        Equation of continuitity & $\bm{\nabla\cdot{J_s}} = -\frac{\partial \rho}{\partial t}$\\

        \hline  
        
      \end{tabular}
    \subsection{Variables}
    %----------------------------------------------------------------------------------------
    %	ARTICLE CONTENTS
    %----------------------------------------------------------------------------------------
    \section{Maxwell' Equations}
    \subsection{Differential Form}

    \begin{tabular}{ l | c }
        \hline
         & Differential Form \\

        \hline			
        Faraday's Law & ${\faradaysLawDiff}$\\
        Ampere's Law & ${\amperesLawDiff}$\\
        
        Gauss' Law, Electric & ${\gaussLawEDiff}$\\
        Gauss' Law, Magnetic & ${\gaussLawMDiff}$\\
        \hline			

        Equation of continuitity & $\bm{\nabla\cdot{J_s}} = -\frac{\partial \rho}{\partial t}$\\

        \hline  
        
      \end{tabular}
    \subsection{Variables}
      These equations are for \textbf{time varying} Electro-Magnetic fields.\\
     
      \noindent$\bm{E}(\bm{r},t)$ is the electric field intensity (V/m)\\
      $\bm{M}(\bm{r},t)$ is the magnetic field intensity (A/m)\\
      $\bm{D}(\bm{r},t)$ is the electric flux density (C/m)\\     
      $\bm{B}(\bm{r},t)$ is the magnetic flux density (Wb/m)\\
      $\bm{J_s}(\bm{r},t)$ is the "impressed" electric surface current density ($\frac{A}{m^2}$)\\
      $\bm{B}(\bm{r},t)$ is the magnetic flux density ($\frac{C}{m^3}$)\\
      $v_g = \frac{\partial\omega}{\partial{k}} = \frac{c}{n} - \frac{ck}{n^2}\frac{\partial{n}}{\partial{k}}$\\
      if the refractive index is constant (not a function of frequency or subsequentally wavelength), 
      $\frac{\partial{n}}{\partial{k}}={0}$ and consequentially 
      $\sin(\frac{w\Delta{t}}{2}) = \pm\frac{c\Delta{t}}{\Delta{x}}\sin(\frac{\overline{k}\Delta{x}}{2})$
      note: if $c\Delta{t}=\Delta{x}$ aka your discrete time interval / discrete position interval = c, you will have no disperion error. Otherwise your system will have a rolling inaccuracy (i.e. $\frac{\Delta{x}}{\Delta{t}}=<c$) and compound the error as the wave propagates.
      \bigskip
      This gives rise to the idea of a \textbf{magic time step} 
      \[
      {\Delta{t}=\frac{c}{\Delta{x}}
      \] 
      imma skrrt

    %\end{equation}

    %\textbf{\nabla \times E} = -\frac{\partial\textbf{B}}{\partial{t}}
    
    

    %------------------------------------------------
    
    \section{Methods}

    %------------------------------------------------
    
    \section{Discussion}
    
    \subsection{Subsection One}
        
    \subsection{Subsection Two}
    
    
    %----------------------------------------------------------------------------------------
    %	REFERENCE LIST
    %----------------------------------------------------------------------------------------
    
    \begin{thebibliography}{99} % Bibliography - this is intentionally simple in this template
    
    \bibitem[Figueredo and Wolf, 2009]{Figueredo:2009dg}
    Figueredo, A.~J. and Wolf, P. S.~A. (2009).
    \newblock Assortative pairing and life history strategy - a cross-cultural
      study.
    \newblock {\em Human Nature}, 20:317--330.
    \section{Equations}

    \end{thebibliography}
    
    %----------------------------------------------------------------------------------------
    
    \end{document}
    